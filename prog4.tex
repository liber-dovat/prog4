\documentclass[a4paper,12pt,oneside]{report}

%----------------------------------------------
% Preambulo

\usepackage{colorful}

\begin{document}

  %----------------------------------------------
  % Titulo

  \title{
    \begin{flushright}
      \Huge  Programaci\'on 4\\
      \LARGE \ColDarkBlue{Un resumen de conceptos}
    \end{flushright}
  } % \title

  \author{\vspace{5cm}
    \\
    \normalsize
    \begin{tabular}{c}
      Compilado por Liber Dovat el \today\\
      Con el aporte de Gonzalo Cedr\'es y Pablo Yaniero\\
      \\
      Versi\'on 1.7\\
      \\
      \href{http://creativecommons.org/licenses/by-nc-sa/3.0/deed.es_CL}{Licenciado bajo CC-by-nc-sa 3.0}
    \end{tabular}
  } % \author

  \date{}
  \maketitle

  \newpage

  %----------------------------------------------
  %Indice

  \pagenumbering{Roman}
  \tableofcontents
  \thispagestyle{plain}

  \newpage

  %----------------------------------------------
  % Contenido

  \pagenumbering{arabic}

  %============================================== Capitulo 1

  \chapter{An\'alisis}
  \thispagestyle{contenido} % no sacar
  \pagestyle{contenido}     % no sacar

    \section{Elementos}
        \paragraph{Orientaci\'on a objetos}
          Puede ser entendida como una forma de pensar basada en abstracciones de conceptos existentes en el mundo real.

        \paragraph{An\'alisis orientado a objetos}
          Consiste en considerar el dominio de la aplicaci\'on y su soluci\'on l\'ogica en t\'erminos de objetos (cosas, conceptos, entidades).

          \subparagraph{Objetivo-1}
            Busca de modelar el dominio del problema para comprender mejor el contexto del problema y para obtener una primera \emph{aproximaci\'on} a la estructura de la soluci\'on.

          \subparagraph{Objetivo-2}
            \begin{itemize}
              \item Modelar el dominio el problema\\
                    Para comprender mejor el contexto del problema.\\
                    Para obtener una primera \emph{aproximaci\'on} a la estructura de la soluci\'on.
              \item Especificar el comportamiento del sistema\\
                    Para contar con una descripci\'on mas precisa de que es lo que se espera del sistema.
            \end{itemize}

        \paragraph{Actividades en el an\'alisis}
          \begin{itemize}
            \item Modelado de dominio\\
                  Consiste en encontrar y describir los objetos (o conceptos) en el dominio de la aplicaci\'on.
            \item Especificaci\'on del comportamiento del sistema\\
                  Consiste en entender a cada caso de uso en t\'erminos de intercambios de mensajes entre los actores y el sistema, y en especificar el comportamiento de cada uno de esos mensajes (Pero sin decir como funcionan).
          \end{itemize}

      %---------------------------------------- seccion 0

    \section{Conceptos b\'asicos}

        \paragraph{Concepto}
          Es una idea, cosa u objeto.

        \paragraph{Clase}
          Es un descriptor de objetos que comparten los mismos atributos,
          operaciones, m\'etodos, relaciones y comportamiento.

        \paragraph{Objeto}
          Es una entidad discreta con limites e identidad bien definidos.\\
          Encapsula estado y comportamiento.\\
          Es una instancia de una clase.

        \paragraph{Identidad}
          Es una propiedad inherente de los objetos de ser distinguibles de
          todos los dem\'as.\\
          Dos objetos son distintos aunque tengan exactamente los mismos
          valores en sus propiedades.

        \paragraph{Atributo}
          Es una descripci\'on de un comportamiento de un tipo especificado
          dentro de una clase.\\
          Puede ser:
            \begin{description}
              \item De instancia:\\
                      cada objeto de esa clase mantiene un valor de ese tipo en
                      forma independiente.
              \item De clase:\\
                      Todos los objetos de esa clase comparten un mismo valor de
                      ese tipo.
            \end{description}

        \paragraph{Operaci\'on}
          Es una \emph{especificaci\'on} de una transformaci\'on o consulta
          que un objeto puede ser llamado a ejecutar.\\
          Tiene asociada un nombre, una lista de par\'ametros y un tipo de
          retorno.

        \paragraph{M\'etodo}
          Es la \emph{implementaci\'on} de una operaci\'on para una
          determinada clase.\\
          Especifica el algoritmo o procedimiento que genera el resultado o
          efecto de la operaci\'on.

        \paragraph{Polimorfismo}
          Es la capacidad de asociar diferentes m\'etodos a la misma operaci\'on.

        \paragraph{Interfaz}
          Es un conjunto de operaciones al que se le aplica un nombre.

        \paragraph{Datatype}
          Es un descriptor de un conjunto de valores que \emph{carecen de identidad}.

        \paragraph{Datavalue}
          Es un valor \'unico que carece de identidad.\\
          Es la instancia de un Datatype.\\
          T\'ipicamente son usados como valores de atributos.

        \paragraph{Referencia}
          Es un valor en tiempo de ejecuci\'on que es \emph{void} o \emph{attached}.
          \begin{itemize}
            \item Si es attached la referencia identifica a un \'unico objeto.
            \item Si es void la referencia no identifica a ning\'un objeto.
          \end{itemize}

        \paragraph{Tipo est\'atico y din\'amico}
          El tipo est\'atico de un objeto es el tipo del cual fue declarada la referencia
          adjunta a \'el; se conoce en tiempo de compilaci\'on.\\
          El tipo din\'amico es el tipo del cual es instancia directa.\\
          En ciertas situaciones ambos tipos coinciden por lo que pierde sentido realizar tal
          distinci\'on.

        \paragraph{Asociaci\'on}
          Describe una relaci\'on sem\'antica entre clasificadores.\\
          es una relaci\'on entre conceptos que indica alguna conexi\'on
          interesante o significativa entre ellos.

        \paragraph{Generalizaci\'on}
          Es una relaci\'on taxon\'omica entre un elemento m\'as general y un elemento m\'as
          espec\'ifico.\\
          El elemento m\'as espec\'ifico es consistente con el m\'as general y
          puede tener informaci\'on adicional.

        \paragraph{Realizaci\'on}
          Es una relaci\'on entre una especificaci\'on y su implementaci\'on.\\
          Por ejemplo, entre una interface y una clase.

        \paragraph{Tipo asociativo}
          Es un elemento que es tanto clase como asociaci\'on.\\
          Agrega propiedades a las asociaciones.

        \paragraph{Roles}
          Especifican el papel que juegan las clases en una asociaci\'on.
          \subparagraph{Cuando utilizar el rol}
            Se utiliza para eliminar la ambig\"uedad cuando existen mas de
            dos asociaciones entre dos clases.

        \paragraph{Agregaci\'on}
          Significa que un elemento es parte de otro.\\
          Existen dos variantes:
          \begin{itemize}
            \item Compartida:\\
                    Es una agregaci\'on en la que las partes no son exclusivas del
                    compuesto.\\
                    Las partes pueden estar incluidas en otros compuestos.
            \item Compuesta:\\
                    Es una agregaci\'on en la que las partes son exclusivas del
                    compuesto.\\
                    Generalmente una acci\'on sobre el compuesto se propaga a las
                    partes (t\'ipicamente en la destrucci\'on).
          \end{itemize}

        \paragraph{Subsumption}
          Es una propiedad que deben cumplir todos los objetos, tambi\'en conocida
          como \emph{intercambiabilidad}.\\
          Un objeto de clase base puede ser sustituido por un objeto de clase derivada
          ( directa o indirecta ).\\
          Por lo tanto: $b:B \wedge B<:A \Longrightarrow b:A$

        \paragraph{Acceso a propiedades}
          La propiedades de una clase tienen aplicadas calificadores de acceso:
          \begin{itemize}
            \item Public:\\
                    Puede se accedida desde cualquier punto desde el cual se tenga
                    visibilidad sobre el objeto.
            \item Private:\\
                    Puede ser accedida solamente desde los m\'etodos de la propia clase.
          \end{itemize}
          Por defecto, los atributos deben ser privados y las operaciones p\'ublicas.

        \paragraph{Descriptores}
          \begin{description}
            \item Full descriptors:\\
                    Es la descripci\'on completa que es necesaria para describir a un
                    objeto.\\
                    Contiene la descripci\'on de todos los atributos, operaciones y
                    asociaciones que el objeto contiene.
            \item Segment descriptor:\\
                    Son los elementos que efectivamente se declaran en un modelo o
                    en el c\'odigo (por ejemplo, clases) y contienen las propiedades
                    heredables que son: Los atributos, las operaciones, los m\'etodos
                    y la participaci\'on en asociaciones.
          \end{description}

        \paragraph{Herencia}
          Es el mecanismo por el cual se permite compartir propiedades entre
          una clase y sus descendientes.

        \paragraph{Clase abstracta}
          Ning\'un objeto puede ser creado directamente a partir de ellas.\\
          No son instanciables.\\
          Existen para que otras hereden las propiedades declaradas por ellas

        \paragraph{Operaci\'on abstracta}
          Es una operaci\'on en una clase sin m\'etodo.

        \paragraph{Modelo}
          Es una abstracci\'on de un sistema desde un punto de vista determinado:
          Funcionalidad, estructura, l\'ogica, estructura f\'isica, etc.

        \paragraph{Relaci\'on entre caso de uso, escenario y DSS}
          Para un caso de uso pueden existir varios escenarios (t\'ipicos y
          alternativos).\\
          Cada escenario de caso de uso se puede representar con un DSS.

        \paragraph{Relaci\'on entre caso de uso y colaboraci\'on}
          Una colaboraci\'on realiza uno o m\'as casos de uso.

    %---------------------------------------- seccion 1

    \section{Modelo de dominio}
        \paragraph{Modelos de las actividades}
          Durante el modelado de dominio, se construye el modelo del dominio, mientras
          que en la especificaci\'on del comportamiento del sistema se completa el
          modelo de casos de uso.

        \paragraph{Invariante}
          Es un predicado que expresa una condici\'on sobre los elementos del
          modelo de dominio y que siempre debe ser verdadero.
          Invariantes habituales:
          \begin{itemize}
            \item Unicidad de atributos (identificaci\'on de instancias):\\
                    Un atributo tiene un valor \'unico dentro del universo de instancias
                    de un mismo tipo (una instancia es identificada por ese valor).
            \item Dominio de atributos:\\
                    El valor de un atributo pertenece a cierto dominio.
            \item Integridad circular:\\
                    no puede existir circularidad en la navegaci\'on.
            \item Atributos calculados:\\
                    El valor de un atributo es calculado a partir de la informaci\'on
                    contenida en el dominio.
            \item Reglas de negocio:\\
                    Invariante que restringe el dominio del problema.
          \end{itemize}

    %---------------------------------------- seccion 2

    \section{OCL}

        \paragraph{Colecci\'on de objetos}
          Ser\'an tratados como meros contenedores de objetos.\\
          Proveer\'an solamente operaciones que permitan administrar los objetos
          contenidos.\\
          En general, las interfaces de diccionario (add, remove, find, member, etc.)
          e iterador (next, etc.) son suficientes.

        \paragraph{Subtipos}
          Se distinguen tres subtipos de collection: Set, Bag y Sequence.\\
          Se corresponden con los tipos abstractos conjunto, bolsa y secuencia
          respectivamente.\\
          Un valor de set es una colecci\'on de elementos donde ellos no se repiten
          y no existe un orden entre ellos.\\
          Un valor de bag es una colecci\'on de elementos donde ellos se pueden repetir
          pero no existe un orden entre ellos.\\
          Un valor de sequence es una colecci\'on de elementos donde ellos se pueden
          repetir y existe un orden entre ellos.

        \paragraph{Ejemplo 1}
          \texttt{\\
             context Etapa inv:\\
             self.Pareja->forAll(p|p.concurso = self.concurso)
          }

        \paragraph{Ejemplo 2}
          \texttt{\\
             context Famoso inv:\\
             self.Pareja->forAll(s1,s2| s1 <> s2 impies s1.concurso <> s2.concurso)
          }

        \paragraph{Ejemplo 3}
          \texttt{\\
             context Aplicacion inv:\\
             Aplicacion.allInstances()->IsUnique(id)
          }

        \paragraph{Ejemplo 4}
          \texttt{\\
            context Empleado inv:\\
            self.trabaja.horas->sum() <= 10
          }

        \paragraph{Ejemplo 5}
          \texttt{\\
            context Vendedor inv:\\
            self.empresa.producto->includes(self.producto)
          }

    %---------------------------------------- seccion 3

    \section{Casos de uso}

      \subsection{Descripci\'on}
        Narra la historia completa (junto a todas sus variantes) de un
        conjunto de actores mientras usan el sistema.

      \subsection{Diagrama de secuencia del sistema}
        Es un artefacto incluido en el modelo de casos de uso, que define e ilustra
        la interacci\'on entre los actores y el sistema en uno o varios escenarios de
        un caso de uso.\\
        Definen la conversaci\'on entre los actores y el sistema, enfoc\'andose
        en los mensajes que el sistema recibe.

        \subsubsection{Elementos del DSS}
          Incluye una instancia representando al sistema, una instancia por cada actor
          que participe y los mensajes enviados entre ellos para el escenario del caso de
          uso que corresponda.
          \begin{itemize}
            \item Incluir una instancia que represente al sistema como una unidad.
            \item Identificar cada actor que participe en el escenario considerado e
                  incluir una instancia para cada uno.
            \item De la descripci\'on del caso de uso, identificar aquellos eventos que
                  los actores generen y sean de inter\'es para el sistema e incluir cada uno
                  de ellos como un mensaje.
            \item Opcionalmente, incluir junto a cada mensaje una descripci\'on.
          \end{itemize}

      \subsection{Contratos de software}
        Especifican declarativamente mediante pre y post condiciones el comportamiento o
        efecto de una operaci\'on.\\
        Existe un contrato de software por cada operaci\'on del DSS.\\
        El Consumidor se compromete a satisfacer la precondici\'on al invocar la operaci\'on.\\
        El Proveedor se compromete a satisfacer la postcondici\'on al finalizar la operaci\'on
        solamente cuando la precondici\'on fue satisfecha al momento de la invocaci\'on.

        \paragraph{Precondici\'on}
          Es a lo que debe acceder el consumidor para obtener el resultado deseado.\\
          Es lo que debe exigir el proveedor para llegar al resultado.\\
          Especifican los valores de los par\'ametros de la operaci\'on y el estado
          del sistema antes de ejecutar la operaci\'on, en t\'erminos de:
            \begin{itemize}
              \item Que un objeto existe.
              \item Que un objeto no existe.
              \item Que un link existe.
              \item Que un link no existe.
              \item Propiedades sobre valores de atributos de objetos.
            \end{itemize}

        \paragraph{Postcondici\'on}
          Es a lo que acceder\'a el consumidor.\\
          Es a lo que se compromete el proveedor.\\
          Especifican el valor de retorno de la operaci\'on y el estado del sistema luego
          de ejecutar la operaci\'on, en t\'erminos de:
            \begin{itemize}
              \item Que un objeto existe.
              \item Que un objeto no existe.
              \item Que un link existe.
              \item Que un link no existe.
              \item Especificar el valor de retorno.
            \end{itemize}

        \paragraph{Memoria del sistema}
          Son los datos que el sistema debe guardar temporalmente mientras se est\'e
          ejecutando un caso de uso.\\
          Representa el estado de la conversaci\'on entre usuario y sistema.

  %============================================== Capitulo 2

  \chapter{Dise\~no}
  \thispagestyle{contenido} % no sacar
  \pagestyle{contenido}     % no sacar

    \section{Elementos}
        \paragraph{Dise\~no orientado a objetos}
          Consiste en definir objetos l\'ogicos (de software) y la forma de
          comunicaci\'on entre ellos para una posterior programaci\'on.

        \paragraph{Arquitectura l\'ogica}
          Conjunto de componentes l\'ogicos relacionados entre si, con
          responsabilidades espec\'ificas.

    %---------------------------------------- seccion 1

    \section{Modelo de dise\~no}
      \subsection{Descripci\'on}
        Es una abstracci\'on de la soluci\'on l\'ogica al problema.

        \paragraph{Controladores}
          Es una clase que implementa las operaciones del sistema.\\
          Se destacan tres tipos de controladores:
          \begin{itemize}
            \item Fachada\\
                    Contiene todas las operaciones del sistema.
            \item Caso de uso\\
                    Contiene todas las operaciones de un caso de uso.
            \item Mini fachada\\
                    Contiene operaciones de varios casos de uso relacionados.
          \end{itemize}

        \paragraph{Criterios GRASP}
          Son criterios que ayudan a resolver el problema de asignar
          responsabilidades.\\
          Sugieren a quien asignar responsabilidades:
          \begin{description}
            \item Expert:\\
                    Responsabilizar a quien tenga la informaci\'on necesaria.
            \item Creator:\\
                    A quien responsabilizar de la creaci\'on de un objeto.
            \item Bajo acoplamiento:\\
                    Evitar que un objeto interact\'ue con demasiados objetos.
            \item Alta cohesi\'on:\\
                    Evitar que un objeto haga demasiado trabajo.
            \item No hables con extra\~nos:\\
                    Asegurarse que un objeto realmente delega trabajo.
            \item Controller:\\
                    A quien responsabilizar de ser el controlador.
          \end{description}

        \paragraph{Clases fuertes}

        \paragraph{Clases d\'ebiles}
          Son clases de menor importancia a cuyas instancias se accede a
          trav\'es de alguna instancia de clase fuerte.

        \paragraph{Manejador}
          Es una clase singleton que contiene el universo de instancias de cierta
          clase y operaciones para la manipulaci\'on de ese universo.\\
          \\
          Las operaciones pueden ser CRUD (create, retrieve, update y delete)
          y operaciones mas complejas que involucren \emph{solamente} las
          instancias fuertes del manejador.\\
          \\
          Permiten que los controladores no dependan entre si para acceder a instancias
          fuertes ni que deban compartir colecciones de esas instancias.

        \paragraph{Interfaces del sistema}
          Buscan quebrar las dependencias entre los elementos de la capa de
          presentaci\'on que invocan operaciones del sistema y los controladores
          de la capa l\'ogica que las implementan.\\
          Contienen las operaciones del sistema que son implementadas por los controladores.

        \paragraph{F\'abrica}
          Es un objeto que tiene la responsabilidad de crear instancias que realicen una
          interfaz determinada.

        \paragraph{C\'omo dise\~nar una colaboraci\'on}
          \begin{itemize}
            \item Definir la estructura y luego generar las diferentes
                  interacciones respetando la estructura.
            \item Definir las interacciones (seg\'un GRAPS) y luego definir la
                  estructura necesaria para que ocurran las interacciones.
          \end{itemize}

        \paragraph{Visibilidad}
          Es la capacidad de un objeto de tener referencia a otro.\\
          Existen cuatro formas b\'asicas de que un objeto A tenga visibilidad sobre otro B:
          \begin{itemize}
            \item Por atributo:\\
                    B es un pseudo atributo de A.
            \item Por par\'ametro:\\
                    B es un par\'ametro de un m\'etodo de A.
            \item Local:\\
                    B es declarado localmente en un m\'etodo de A.
            \item Global:\\
                    B es visible en forma global.
          \end{itemize}

    %---------------------------------------- seccion 2

    \section{Diagrama de comunicaci\'on}
        \paragraph{Descripci\'on}
          Son artefactos mediante los cuales se expresar\'an las
          interacciones.

    %---------------------------------------- seccion 3

    \section{Patrones de dise\~no}
      \subsection{Descripci\'on}
        Explica un dise\~no general que se aplica a un problema de dise\~no a
        objetos.

        \paragraph{Strategy}
          Busca definir una familia de algoritmos, encapsularlos y hacerlos
          intercambiables.\\
          Esto  permite que el algoritmo var\'ie dependiendo del cliente que lo utiliza.

        \paragraph{Composite}
          Componer objetos en estructuras arborescentes para representar jerarqu\'ia de
          objetos compuestos y tratar uniformemente los mismos.

        \paragraph{Singleton}
          Asegurar que una clase tenga una sola instancia y proveer un acceso global a ella.

        \paragraph{State}
          Permite que un objeto var\'ie su comportamiento cuando su estado interno
          cambie.\\
          El objeto parecer\'a haber cambiado de clase.

        \paragraph{Proxy}
        \paragraph{Template method}
        \paragraph{Observer}
        \paragraph{Adapter}

  %============================================== Capitulo 3

  \chapter{Implementaci\'on}
  \thispagestyle{contenido} % no sacar
  \pagestyle{contenido}     % no sacar

    \section{Elementos}
        \paragraph{Implementaci\'on orientada a objetos}
        Consiste en codificar en un lenguaje de programaci\'on orientada a
        objetos los mecanismos definidos en el dise\~no.

    %---------------------------------------- seccion 1

    \section{Ejemplos de c\'odigo}

%       \lstset{
%         language         = C++,
%         tabsize          = 3,
%         basicstyle       = \small,
%         %keywordstyle    = \color{blue}\bfseries,
%         %stringstyle     = \color{red},
%         %commentstyle    = \color{cyan},
%         showstringspaces = false
%       }

      \lstset{
        language         = C++,
        breaklines       = true,
        breakindent      = 30pt,                                    % chr
        prebreak         = \mbox{\hspace{5pt}\tiny$\searrow$},      % chr
        postbreak        = \mbox{\ColDarkBlue{\tiny$\rightarrow$}}, % chr
        tabsize          = 3,
        basicstyle       = \small,
        keywordstyle     = \color{blue}\bfseries,
        stringstyle      = \color{red},
        commentstyle     = \color{gray},
        showstringspaces = false
      }

      \paragraph{Atributo}
        \texttt{
          \lstinputlisting{atributo.cc}
        }

      \paragraph{Singleton}
        \texttt{
          \lstinputlisting{singleton.cc}
        }

      \paragraph{Dynamic cast}
        \texttt{
          \lstinputlisting{dynamic_cast.cc}
        }

      \paragraph{Interface}
        \texttt{
          \lstinputlisting{interface.cc}
        }

      \paragraph{Subclase}
        \texttt{
          \lstinputlisting{subclase.cc}
        }

      \paragraph{Iterador}
        \texttt{
          \lstinputlisting{iterador.cc}
        }

\end{document}
